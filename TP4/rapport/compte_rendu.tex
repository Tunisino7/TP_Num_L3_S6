\documentclass{../latex/TP}

\begin{document}

\entete{Votre nom}{Numéro étudiant}{LU3PY126 FOAD}{TP 4 : Équation de Schrödinger}

\begin{multicols}{2}
\tableofcontents
\end{multicols}
\clearpage

\section*{Introduction}

\begin{paracol}{2}
\textbf{Objectifs du TP :}
\begin{itemize}[label=\ding{212}]
    \item Résoudre l'équation de Schrödinger 1D par différences finies
    \item Diagonaliser des matrices tridiagonales
    \item Étudier trois potentiels : puits infini, harmonique, double puits
    \item Comparer solutions numériques et théoriques
    \item Analyser l'effet tunnel dans le double puits
\end{itemize}

\textbf{Organisation du code :}
\begin{itemize}[label=\ding{212}]
    \item Fichiers Python : \texttt{../python/q01\_q02.py} à \texttt{q14.py}
    \item Graphiques : \texttt{../figures/*.pdf}
    \item Bibliothèques : numpy, scipy, matplotlib
\end{itemize}

\switchcolumn

\textbf{Méthode numérique :}\\
On discrétise l'équation de Schrödinger stationnaire :
\[
-\frac{\hbar^2}{2m}\frac{d^2\psi}{dx^2} + V(x)\psi = E\psi
\]

La dérivée seconde devient :
\[
\frac{d^2\psi}{dx^2} \approx \frac{\psi_{i+1} - 2\psi_i + \psi_{i-1}}{\delta x^2}
\]

On obtient un problème de valeurs propres :
\[
\mathbf{H}\Psi = E\Psi
\]

avec $\mathbf{H}$ matrice tridiagonale.

\textbf{Exécution :}
\begin{verbatim}
python run_all.py
\end{verbatim}

\end{paracol}
\clearpage

% ==================== QUESTIONS 1-2 ====================
\section{Définition des potentiels}

\subsection{Énoncé}
Définir et tracer trois potentiels sur $[-L/2, L/2]$ avec $L=5$, $n=100$ :
\begin{itemize}
    \item Puits infini : $V(x) = 0$
    \item Oscillateur harmonique : $V(x) = \frac{1}{2}x^2$
    \item Double puits : polynôme degré 4 avec racines en $r_1, r_2, r_3, r_4$
\end{itemize}

\subsection{Code}
\inputminted[fontsize=\small, linenos, breaklines]{python}{../python/q01_q02.py}
\clearpage

\subsection{Résultats}
\begin{figure}[H]
    \centering
    \includegraphics[width=\textwidth]{../figures/q01_q02.pdf}
\end{figure}

\textbf{Commentaire :}\\
Les trois potentiels sont correctement définis. Le double puits présente deux minima symétriques séparés par une barrière centrale.
\clearpage

% ==================== QUESTION 3 ====================
\section{Conditions aux limites}

\subsection{Énoncé}
Interpréter physiquement les conditions $\psi(-L/2) = \psi(L/2) = 0$.

\subsection{Code}
\inputminted[fontsize=\small, linenos, breaklines]{python}{../python/q03.py}
\clearpage

\subsection{Discussion}
Les conditions $\psi(\pm L/2) = 0$ signifient que la particule ne peut pas exister aux bords du domaine. Cela équivaut à placer des murs infinis ($V \to \infty$) en $x = \pm L/2$.

\textbf{Conséquences :}
\begin{itemize}
    \item La particule est confinée dans $[-L/2, L/2]$
    \item États liés uniquement
    \item Spectre discret d'énergies
    \item Impact sur la précision : si $L$ est trop petit, les fonctions d'onde sont tronquées artificiellement
\end{itemize}
\clearpage

% ==================== QUESTIONS 4-5 ====================
\section{Construction et diagonalisation de l'hamiltonien}

\subsection{Énoncé}
\textbf{Q4 :} Construire la matrice hamiltonienne tridiagonale.

\textbf{Q5 :} Diagonaliser avec \texttt{scipy.linalg.eigh\_tridiagonal}.

\subsection{Code}
\inputminted[fontsize=\small, linenos, breaklines]{python}{../python/q04_q05.py}
\clearpage

\subsection{Méthode}
L'hamiltonien est une matrice tridiagonale $(n-2) \times (n-2)$ :
\begin{itemize}
    \item Diagonale : $d_i = \frac{2}{\delta x^2} + V_i$
    \item Sous/sur-diagonale : $e_i = -\frac{1}{\delta x^2}$
\end{itemize}

La fonction \texttt{scipy.linalg.eigh\_tridiagonal} exploite la structure tridiagonale pour une diagonalisation efficace en $O(n^2)$ au lieu de $O(n^3)$.
\clearpage

% ==================== QUESTION 6 ====================
\section{Puits infini : comparaison des énergies}

\subsection{Énoncé}
Comparer les énergies numériques avec la solution théorique :
\[
E_p = \frac{\pi^2(p+1)^2}{L^2}
\]

\subsection{Code}
\inputminted[fontsize=\small, linenos, breaklines]{python}{../python/q06.py}
\clearpage

\subsection{Résultats}
\begin{figure}[H]
    \centering
    \includegraphics[width=\textwidth]{../figures/q06_puits_energies_n100_L5.pdf}
\end{figure}

\textbf{Commentaire :}\\
L'accord est excellent pour les bas niveaux. Pour $p$ élevé, l'écart augmente car la longueur d'onde devient comparable à $\delta x$, violant la condition de discrétisation fine.
\clearpage

% ==================== QUESTION 7 ====================
\section{Normalisation et visualisation des fonctions d'onde}

\subsection{Énoncé}
Normaliser les fonctions d'onde et tracer $|\psi_p(x)|^2$ décalées de $E_p$ pour $p=0,1,2$.

\subsection{Code}
\inputminted[fontsize=\small, linenos, breaklines]{python}{../python/q07.py}
\clearpage

\subsection{Résultats}
\begin{figure}[H]
    \centering
    \includegraphics[width=\textwidth]{../figures/q07_puits_fonctions_n100_L5.pdf}
\end{figure}

\textbf{Commentaire :}\\
Les densités de probabilité montrent :
\begin{itemize}
    \item $p=0$ : pas de nœud, maximum au centre
    \item $p=1$ : un nœud au centre
    \item $p=2$ : deux nœuds
\end{itemize}
Le nombre de nœuds est égal à $p$ (nombre quantique).
\clearpage

% ==================== QUESTION 8 ====================
\section{Comparaison détaillée numérique vs théorique}

\subsection{Énoncé}
Comparer $\psi_{num}$ et $\psi_{theo}$ pour $p=1$ (bon accord) et $p=55$ (mauvais accord).

\subsection{Code}
\inputminted[fontsize=\small, linenos, breaklines]{python}{../python/q08.py}
\clearpage

\subsection{Résultats}
\columnratio{0.5}
\begin{paracol}{2}
\textbf{Cas $p=1$ :}\\
Accord excellent. La fonction d'onde a une longueur d'onde grande devant $\delta x$.
\begin{figure}[H]
    \centering
    \includegraphics[width=\columnwidth]{../figures/q08_puits_comparaison_p1_n100_L5.pdf}
\end{figure}
\switchcolumn
\textbf{Cas $p=55$ :}\\
Accord médiocre. Les oscillations rapides ne sont pas résolues : il faut au moins quelques points par oscillation. Augmenter $n$ résoudrait le problème.
\begin{figure}[H]
    \centering
    \includegraphics[width=\columnwidth]{../figures/q08_puits_comparaison_p55_n100_L5.pdf}
\end{figure}
\end{paracol}
\clearpage

% ==================== QUESTION 9 ====================
\section{Oscillateur harmonique : L=5}

\subsection{Énoncé}
Étudier $V(x) = \frac{1}{2}x^2$ avec $L=5$, $n=100$. Comparer avec :
\[
E_p = \left(p + \frac{1}{2}\right)\sqrt{2}
\]

\subsection{Code}
\inputminted[fontsize=\small, linenos, breaklines]{python}{../python/q09.py}
\clearpage

\subsection{Résultats}
\begin{figure}[H]
    \centering
    \includegraphics[width=\textwidth]{../figures/q09_harmonique_energies_n100_L5.pdf}
\end{figure}

\textbf{Commentaire :}\\
L'accord est bon pour les bas niveaux mais se dégrade rapidement. Raison : pour $E$ élevée, la fonction d'onde s'étend loin et est tronquée par les conditions $\psi(\pm L/2) = 0$. La particule "sent" les murs artificiels.
\clearpage

% ==================== QUESTION 10 ====================
\section{Oscillateur harmonique : L=20}

\subsection{Énoncé}
Refaire Q9 avec $L=20$ pour étudier l'influence de $L$.

\subsection{Code}
\inputminted[fontsize=\small, linenos, breaklines]{python}{../python/q10.py}
\clearpage

\subsection{Résultats}
\begin{figure}[H]
    \centering
    \includegraphics[width=\textwidth]{../figures/q10_harmonique_energies_n100_L20.pdf}
\end{figure}

\textbf{Commentaire :}\\
Avec $L=20$, l'accord est bien meilleur. Les conditions aux limites perturbent moins car la fonction d'onde décroît exponentiellement loin du centre. Inconvénient : $\delta x = L/(n-1)$ augmente donc la discrétisation est plus grossière. Compromis : augmenter $n$ en même temps que $L$.
\clearpage

% ==================== QUESTION 11 ====================
\section{Fonctions d'onde de l'oscillateur harmonique}

\subsection{Énoncé}
Tracer $V(x)$ et $|\psi_p(x)|^2$ décalées de $E_p$ pour $p=0,1,2$ avec $L=20$, $n=100$.

\subsection{Code}
\inputminted[fontsize=\small, linenos, breaklines]{python}{../python/q11.py}
\clearpage

\subsection{Résultats}
\begin{figure}[H]
    \centering
    \includegraphics[width=\textwidth]{../figures/q11_harmonique_fonctions_n100_L20.pdf}
\end{figure}

\textbf{Commentaire :}\\
\begin{itemize}
    \item $p=0$ : gaussienne centrée (pas de nœud)
    \item $p=1$ : un nœud au centre, densité nulle en $x=0$
    \item $p=2$ : deux nœuds, trois lobes
\end{itemize}
Les maxima de densité correspondent aux points de retour classiques où $V(x) = E_p$ (vitesse classique nulle).
\clearpage

% ==================== QUESTION 12 ====================
\section{Double puits symétrique (a=1)}

\subsection{Énoncé}
Étudier le potentiel double puits avec $a=1$, $r_1=-2$, $r_2=-0.5$, $r_3=0.5$, $r_4=2$, $L=20$, $n=1000$.

\subsection{Code}
\inputminted[fontsize=\small, linenos, breaklines]{python}{../python/q12.py}
\clearpage

\subsection{Résultats}
\begin{figure}[H]
    \centering
    \includegraphics[width=\textwidth]{../figures/q12_double_puits_sym_a1.pdf}
\end{figure}

\textbf{Commentaire :}\\
Le potentiel est symétrique par rapport à $x=0$. Les états propres sont soit symétriques ($\psi(-x) = \psi(x)$) soit antisymétriques ($\psi(-x) = -\psi(x)$).

\textbf{Doublets :}
\begin{itemize}
    \item $\psi_0$ : état fondamental symétrique
    \item $\psi_1$ : premier excité antisymétrique, nœud en $x=0$
    \item Faible écart $\Delta E = E_1 - E_0$ : effet tunnel significatif
\end{itemize}

La barrière centrale (faible avec $a=1$) permet un fort couplage entre les deux puits.
\clearpage

% ==================== QUESTION 13 ====================
\section{Double puits symétrique profond (a=400)}

\subsection{Énoncé}
Refaire Q12 avec $a=400$ pour augmenter la hauteur de la barrière.

\subsection{Code}
\inputminted[fontsize=\small, linenos, breaklines]{python}{../python/q13.py}
\clearpage

\subsection{Résultats}
\begin{figure}[H]
    \centering
    \includegraphics[width=\textwidth]{../figures/q13_double_puits_sym_a400.pdf}
\end{figure}

\textbf{Commentaire :}\\
Avec $a=400$, la barrière est bien plus haute. L'effet tunnel est fortement supprimé :
\begin{itemize}
    \item Les fonctions d'onde sont très localisées dans chaque puits
    \item Amplitude quasi-nulle dans la région centrale
    \item Doublets presque parfaitement dégénérés : $\Delta E \approx 0$
    \item Limite : deux puits découplés avec spectres indépendants
\end{itemize}
\clearpage

% ==================== QUESTION 14 ====================
\section{Double puits asymétrique}

\subsection{Énoncé}
Refaire Q12 avec $r_3=0$ au lieu de $0.5$ pour briser la symétrie.

\subsection{Code}
\inputminted[fontsize=\small, linenos, breaklines]{python}{../python/q14.py}
\clearpage

\subsection{Résultats}
\begin{figure}[H]
    \centering
    \includegraphics[width=\textwidth]{../figures/q14_double_puits_asym_a1.pdf}
\end{figure}

\textbf{Commentaire :}\\
La symétrie est brisée : le puits gauche est plus large que le droit.

\textbf{Conséquences :}
\begin{itemize}
    \item Les fonctions d'onde ne sont plus symétriques/antisymétriques
    \item $\psi_0$ : localisée principalement dans le puits gauche (plus large, énergie plus basse)
    \item $\psi_1$ : localisée principalement dans le puits droit
    \item Plus de doublets quasi-dégénérés : levée de dégénérescence par brisure de symétrie
    \item Chaque puits impose sa propre structure de niveaux
\end{itemize}
\clearpage

% ==================== CONCLUSION ====================
\section*{Conclusion}

Ce TP a permis de :
\begin{enumerate}
    \item Implémenter la méthode des différences finies pour résoudre l'équation de Schrödinger 1D
    \item Diagonaliser efficacement des matrices tridiagonales avec \texttt{scipy}
    \item Comparer solutions numériques et analytiques pour le puits infini
    \item Comprendre l'impact des paramètres $L$ et $n$ sur la précision
    \item Étudier l'effet tunnel dans le double puits
    \item Observer la levée de dégénérescence par brisure de symétrie
\end{enumerate}

\textbf{Limites de la méthode :}
\begin{itemize}
    \item Conditions aux limites artificielles : nécessite $L$ grand pour les potentiels non-confinés
    \item Discrétisation spatiale : nécessite $\delta x$ petit devant la longueur d'onde
    \item Compromis $L$ vs $n$ : augmenter $L$ améliore les conditions aux limites mais augmente $\delta x$ à $n$ fixé
\end{itemize}

\textbf{Extensions possibles :}
\begin{itemize}
    \item Potentiels 2D ou 3D (matrices plus complexes)
    \item Dépendance temporelle : méthode Crank-Nicolson
    \item Potentiels aléatoires : localisation d'Anderson
    \item Systèmes à N corps : problème combinatoire
\end{itemize}

\end{document}
